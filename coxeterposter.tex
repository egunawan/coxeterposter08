% Template file for an a0 landscape poster.
% Written by Graeme, 2001-03 based on Norman's original microlensing
% poster.
%
% See discussion and documentation at
% <http://www.astro.gla.ac.uk/users/norman/docs/posters/> 
%
% $Id: poster-template-landscape.tex,v 1.2 2002/12/03 11:25:46 norman Exp $


% Default mode is landscape, which is what we want, however dvips and
% a0poster do not quite do the right thing, so we end up with text in
% landscape style (wide and short) down a portrait page (narrow and
% long). Printing this onto the a0 printer chops the right hand edge.
% However, 'psnup' can save the day, reorienting the text so that the
% poster prints lengthways down an a0 portrait bounding box.
%
% 'psnup -w85cm -h119cm -f poster_from_dvips.ps poster_in_landscape.ps'

\documentclass[a0]{a0poster}
% You might find the 'draft' option to a0 poster useful if you have
% lots of graphics, because they can take some time to process and
% display. (\documentclass[a0,draft]{a0poster})

\pagestyle{empty}
\setcounter{secnumdepth}{0}

% The textpos package is necessary to position textblocks at arbitary 
% places on the page.
\usepackage[absolute]{textpos}

% Graphics to include graphics. Times is nice on posters, but you
% might want to switch it off and go for CMR fonts.
\usepackage{graphics,wrapfig,times}

\usepackage{amsmath, amsthm, amssymb, graphicx, color, amsfonts}%coxeter group

% These colours are tried and tested for titles and headers. Don't
% over use color!
\usepackage{color}
\definecolor{DarkBlue}{rgb}{0.1,0.1,0.5}
\definecolor{Red}{rgb}{0.9,0.0,0.1}
\definecolor{DarkGreen}{rgb}{.2, .5, .1}
\definecolor{Purple}{rgb}{.5, 0, 0.5}

% see documentation for a0poster class for the size options here
\let\Textsize\normalsize
\def\Head#1{\noindent{\LARGE\color{DarkBlue} #1}\bigskip}
\def\LHead#1{\noindent{\LARGE\color{DarkBlue} #1}\bigskip}
\def\Subhead#1{\noindent{\large\color{DarkBlue} #1}\bigskip}
\def\Title#1{\noindent{\VeryHuge\color{Purple} #1}}

\newcommand\NN{\mathbb N} %coxeter group
\def\I{\mathcal I}

% Set up the grid
%
% Note that [40mm,40mm] is the margin round the edge of the page --
% it is _not_ the grid size. That is always defined as 
% PAGE_WIDTH/HGRID and PAGE_HEIGHT/VGRID. In this case we use
% 23 x 12. This gives us three columns of width 7 boxes, with a gap of
% width 1 in between them. 12 vertical boxes is a good number to work
% with.
%
% Note however that texblocks can be positioned fractionally as well,
% so really any convenient grid size can be used.
%
\TPGrid[40mm,40mm]{23}{12}      % 3 cols of width 7, plus 2 gaps width 1
%\TPGrid[40mm,40mm]{48}{36} 	% Event Organizers suggested 36 (wide) x 48 (wide): 3'x4'

\parindent=0pt
\parskip=0.5\baselineskip

\begin{document}

% Understanding textblocks is the key to being able to do a poster in
% LaTeX. In
%
%    \begin{textblock}{wid}(x,y)
%    ...
%    \end{textblock}
%
% the first argument gives the block width in units of the grid
% cells specified above in \TPGrid; the second gives the (x,y)
% position on the grid, with the y axis pointing down.

% You will have to do a lot of previewing to get everything in the 
% right place.

% This gives good title positioning for a portrait poster.
% Watch out for hyphenation in titles - LaTeX will do it
% but it looks awful.
\begin{textblock}{23}(0,0)
\begin{center}
\Title{Involutions in Coxeter Groups}
%\end{center}
%\end{textblock}

%\begin{textblock}{8}(0,1.0)
%\begin{center}
\LHead{Anna Boatwright,
Emily Gunawan,
Jennifer Koonz }

\LHead{Faculty Advisor: Ruth Haas, Smith College Department of Mathematics and Statistics} 

\LHead{The Center for Women in Mathematics at Smith College}
%}

\end{center}
\end{textblock}

% Uni logo in the top right corner. A&A in the bottom left. Gives a
% good visual balance, but you may want to change this depending upon
% the graphics that are in your poster.
\begin{textblock}{2}(1.8,0)
\includegraphics[width=4in]{smith.jpg}
%\includegraphics{/usr/local/share/images/AandA.epsf}
\end{textblock}

\begin{textblock}{2}(19,0)
\includegraphics[width=4in]{nsf.jpg}
%\resizebox{2\TPHorizModule}{!}{\includegraphics{/usr/local/share/images/GUVIu/GUVIu.eps}}
\end{textblock}


\begin{textblock}{7}(0,2.2)
\hrule\medskip

\Head{Abstract}

{\small A Coxeter group is a group which is generated by
involutions. They are often used to study geometrical symmetries. Two finite Coxeter groups were studied, those of type
$B_n$ and those of type $D_n$. The research group, led by Professor
Ruth Haas, was primarily interested in the conjugacy classes of the involutions of each group and the
relationships between these conjugacy classes. The elements of each involution conjugacy class of $B_n$ were explicitly determined.  Formulae were found to count the
order of each involution conjugacy class of $B_n$ and to count the number of involution 
conjugacy classes in $B_n$. A relationship between the involution conjugacy classes of $B_n$ was determined.  The involution conjugacy classes in the subgroup $D_n$ were studied. }

\bigskip

\hrule
\Head{The Coxeter Group $B_n$}

For any $n \in \mathbb{N}$, consider the set 
$\{ 1, 2, \hdots ,n,  -1, -2, \hdots, -n \}.$
The Coxeter group $B_n$ is the group generated by the permutations $s_1, s_2, \hdots, s_n$ where
 \begin{eqnarray*}
 s_1 &=& (1\, 2)(-1 \, -2) \\
 s_2 &=& (2\, 3)(-2 \, -3) \\
& \vdots & \\
 s_{n-1} &=& ((n-1) \, n)(-(n-1) \, -n) \\
 s_n &=& (1 \, -1) 
 \end{eqnarray*}
 
{\color{DarkBlue}{ \large Properties of $B_n$ : }}
\begin{itemize}
\item Note that each generator $s_i$ is an involution. 

\item A defining characteristic of $B_n$ is that for any element $\sigma$ of $B_n$ and any \\
$i,j \in \{1, \hdots, n, -1, \hdots, -n \}$, if $\sigma (i) =j$, then $\sigma( -i) = -j$. 
\end{itemize}
\bigskip
\hrule
\Head{Properties of Involutions }

{\bf Lemma 1:}
If $w$ is an involution and $s$ is any element in a group $G$, then $sws^{-1}$ is an involution. \\
\textit{Proof.} Then $(sws^{-1})(sws^{-1})= sws^{-1}sws^{-1} =  swws^{-1} = ss^{-1} = e$. 

{\bf Lemma 2:}
Suppose $w$ and $s$ are involutions in a group $G$. If $sws^{-1} = w$, then $sw$ is an involution. \\
\textit{Proof.} Then $swsw = (sws^{-1})w = ww = e$. 

{\bf Theorem:}
For a group $G$ with generating set $S$, all involutions of $G$ can be generated by repeated application of Lemma 1 and Lemma 2 using $S$, starting with the identity.
\bigskip
\hrule

\Head{Involution Posets of $B_n$}
\begin{figure}
\begin{center}
\includegraphics[width=2in]{B2.pdf}
\includegraphics[width=2in]{B3.jpg}
\includegraphics[width=2in]{B4.pdf}
\includegraphics[width=4in]{B5.pdf}
\caption{(From left to right) The involution posets for $B_2$, $B_3$, $B_4$, and $B_5$.} 
\end{center}
\end{figure}

\end{textblock}

%%%%%%
%%%%%%%%% 2nd column begins
\begin{textblock}{7}(8,2.2)
\hrule
%%%
\Head{The Conjugacy Classes of $B_n$}

{\bf Definition:} Elements of $B_n$ of the form $(a \, -a)$ will be called $\alpha$-cycles, and elements of the form $(a \, b) (-a \, -b)$ (where $b \neq \pm a$) will be called $\beta$-cycles. We say that two involutions $\sigma$ and $\tau$ of $B_n$ have the same {\it cycle type} if they consist of the same number of $\alpha$-cycles and the same number of $\beta$- cycles. 

\noindent {\bf Theorem 1:} Let $\sigma$ and $\tau$ be involutions in $B_n$. Then $\sigma$ is conjugate to $\tau$ if and only if $\sigma$ and $\tau$ have the same cycle type.

{\bf Notation:} Let $C$ be a conjugacy class in $B_n$. Then every element of $C$ is composed of $s$ $\alpha$-cycles and $t$ $\beta$-cycles, and we can denote the class $C$ by $[s,t]$. 

{\bf Theorem 2:} In $B_n$,  the number of elements in each conjugacy class $[s,t]$ is given by 
\[ {\frac{n!}{(n-2t)!t!}}   {{n-t}\choose{s}} .  \]


{\bf Theorem 3:} For each $n$, the number of conjugacy classes in $B_n$ is given by
\[\sum_{k=0}^n \left( \left\lfloor \frac{k}{2} \right\rfloor + 1\right) . \]

 \hrule
 
 %%%%%%%%%
\Head{ $B_n$ Supergraphs}

Define the \textit{Supergraph} of $B_n$ to be the graph where each conjugacy class $[s,t]$ of $B_n$ is a vertex, and an edge between two vertices, $[s,t]$ and $[s',t']$, indicates that there exists some $\sigma \in [s,t]$ and a generator $s_i$ of $B_n$ such that $s_i \sigma s_i^{-1} = \sigma$ and $s_i \sigma \in [s',t']$. 

\begin{figure}
\begin{center}
\includegraphics[width=2in]{SupergraphB2.pdf}
\hspace{.5in}
\includegraphics[width=2in]{SupergraphB3.pdf}
\caption{The Supergraphs of for $B_2$ and $B_3$.}
\end{center}
\end{figure}

\begin{figure}
\begin{center}
\includegraphics[width=2.5in]{SupergraphB7.pdf}
\hspace{.3in}
\includegraphics[width=7in]{SupergraphB7_Cart.pdf}
\caption{The Supergraph of for $B_7$, also shown mapped onto the Cartesian plane.}
\end{center}
\end{figure}


\end{textblock}

%third column begins here
\begin{textblock}{7}(16,2.2)
\hrule
\Head{Properties of the $B_n$ Supergraphs}

Notice that the Supergraph of $B_n$ always contains the Supergraphs of $B_k$ for all $k \leq n$ as subgraphs.\\

{\bf Theorem 4:} The conjugacy class $[s,t]$ is adjacent to the conjugacy class $[s',t']$ in the Supergraph of $B_n$ iff one of the following hold:
\begin{enumerate}
\item $s+1 = s'$ and $t=t'$ (or $s = s'+1$ and $t = t'$)
\item $s = s'$ and $t+1 = t'$ (or $s = s'$ and $t = t'+1$)
\item $s+2 = s'$ and $t-1 = t'$ (or $s = s'+2$ and $t = t'-1$)
\end{enumerate}

\hrule

 \Head{The Coxeter Group $D_n$}
 
Coxeter Groups of type $D_n$ are generated by the elements
\begin{eqnarray*}
 s_1 &=& (1\, 2)(-1 \, -2) \\
 s_2 &=& (2\, 3)(-2 \, -3) \\
& \vdots & \\
 s_{n-1} &=& ((n-1) \, n)(-(n-1) \, -n) \\
 s_n &=& (1 \, -2) (-1 \, 2) 
 \end{eqnarray*}
 %
{\color{DarkBlue}{ \large Properties of $D_n$: }}
\begin{itemize}
\item $D_n$ is a subgroup of $B_n$ 
\item Every element of $D_n$, when written in bottom row notation, must have an even amount of negative signs. For example  $[3 \, -4 \, 2 \, -1 \, 5]$ is an element of $D_5$ but $[1 \, -2 \, 3 \, 4 \, 5]$ is not. 
\end{itemize}

\hrule

\Head{Involution Posets for $D_n$}

\begin{figure}
\begin{center}
\includegraphics[width=2in]{D2.pdf}
\hspace{1in}
\includegraphics[width=2in]{D3.pdf}
\caption{The involution poset for $D_2$ (left) and the involution poset for $D_3$ (right).} 
\end{center}
\end{figure}


\hrule

\Head{Future Work}

In the future, we plan to study the conjugacy classes of $D_n$ by:
\begin{itemize}
\item Examining the involution posets for $D_n$
\item Determining the Supergraphs for $D_n$
\item Determining which properties of the conjugacy classes of $B_n$ hold in $D_n$
\end{itemize}

\hrule
\Head{Acknowledgements}

We would like to thank Ruth Haas, the Smith College Department of Mathematics and Statistics, the Center for Women in Math at Smith College, and the National Science Foundation (NSF Grant DMS-0602110). 
  
\end{textblock}


\end{document}
